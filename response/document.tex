\documentclass[onecolumn, 11pt, draftclsnofoot]{IEEEtran}

\usepackage{epsfig,graphics,epsf,amsfonts,color,mathrsfs,amssymb,bm,caption,algorithm,algorithmic}
\usepackage{url}

\usepackage{amsmath}
\newcommand{\bsb}{\boldsymbol}
\newcommand{\mb}{\mathbf}
\newcommand{\mc}{\mathcal}
\newcommand{\mexpc}{\mathrm{E}}
\newcommand{\ve}{\mathrm{vec}}
\newcommand{\toep}{\mathcal{T}}
\newcommand{\tr}{\mathrm{tr}}
\newcommand{\tred}{\color{red}}
\textheight=9.2in
\DeclareCaptionLabelFormat{lc}{\MakeLowercase{#1}~#2}
\captionsetup{labelfont=sc,labelformat=lc}
\renewcommand{\figurename}{Figure}
\renewcommand{\theequation}{R\arabic{equation}}

\usepackage[numbers]{natbib}

%\hyphenation{op-tical net-works semi-conduc-tor}
\begin{document}
The authors appreciate the helpful comments and suggestions from both reviewers. 
Throughout this response letter, we will follow the following notational
rules for citations and references:
\begin{itemize}
  \item Citations: references in this response letter are cited with prefix
  ``R'' (e.g. [R 1]) whereas references in the original manuscript are cited
  without prefix (e.g. [1]).
  \item Equations: equations in this response letter are referred to as
  ``Eq.~(R1)''  while those in the original manuscript as ``Eq.~(1)''.
\end{itemize}

\begin{center}
  {\LARGE \textbf{Authors' Response to Reviewer 1}}
\end{center}

% ~\citep[R][]{cheng2013thermal}
% \begin{align}
%   sfsdf
%   \label{eq:test}
% \end{align}
% \eqref{eq:test}

%%%%%%%%%%%%%%%%%%%%%%%%%%%%%%%%%%%%%%%%%%%%%%%%%%%%%%%%%%%%%%%%
\noindent
\emph{1. The signal model needs further work: p is reported as “index” before
(1) and something to be demodulated in (3); in (1), which is the transmit
vector/symbol? }

\noindent \textbf{Authors' response:}
Thank you for the question. 

\vspace{0.5cm}

%%%%%%%%%%%%%%%%%%%%%%%%%%%%%%%%%%%%%%%%%%%%%%%%%%%%%%%%%%%%%%%%%
\noindent
\emph{2. Some more not defined expressions/symbols/variables: in (2), H\_a
(first term); CSCG.}

\noindent \textbf{Authors' response:}
Thank you.

\vspace{0.5cm}

%%%%%%%%%%%%%%%%%%%%%%%%%%%%%%%%%%%%%%%%%%%%%%%%%%%%%%%%%%%%%%%%%%
\noindent
\emph{3. Proposition 1: the proof is abridged in excess. For instance, the
derivation of (9) is hard to understand.}

\noindent \textbf{Authors' response:}
Thank you for the comment.

\vspace{0.5cm}

%%%%%%%%%%%%%%%%%%%%%%%%%%%%%%%%%%%%%%%%%%%%%%%%%%%%%%%%%%%%%%%%
\noindent
\emph{4. In Section III.B, the method has not been presented with sufficient
detail. Why do you introduce x variables? What is the role of “flow” and
“distance” matrices? How do you derive (13)?}

\noindent \textbf{Authors' response:}

\vspace{0.5cm}

%%%%%%%%%%%%%%%%%%%%%%%%%%%%%%%%%%%%%%%%%%%%%%%%%%%%%%%%%%%%%%%%
\noindent
\emph{5. The proposed solving method based on ILS has not been sufficiently
explained. Among others, the method is apparently distributed but in (11) the
authors have a global function.}

\noindent \textbf{Authors' response:}

\vspace{0.5cm}

%%%%%%%%%%%%%%%%%%%%%%%%%%%%%%%%%%%%%%%%%%%%%%%%%%%%%%%%%%%%%%%%

As per the other concerns, the assumption of two cooperative RRHs

The reviewer also questioned the consistency of our simulation result.

Once again, we thank the reviewer for the review and suggestions that helped to
significantly improve our revised manuscript.



\newpage
\begin{center}
{\LARGE \textbf{Authors' Response to Reviewer 2}}
\end{center}

%%%%%%%%%%%%%%%%%%%%%%%%%%%%%%%%%%%%%%%%%%%%%%%%%%%%%%%%%%%%%%%%%%%%%
\noindent
\emph{None. I suggest to accept the manuscript for publication in IEEE
communications Letters.}

\noindent \textbf{Authors' response:}
Thank you for your recommendation.

\vspace{0.5cm}

 %\newpage
\bibliographystyle{IEEEtranN}
\bibliography{IEEEabrv,refs.bib}
\end{document} 