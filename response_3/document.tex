\documentclass[onecolumn, 11pt, draftclsnofoot]{IEEEtran}

\usepackage{epsfig,graphics,epsf,amsfonts,color,mathrsfs,amssymb,bm,caption,algorithm,algorithmic}
\usepackage{url}

\usepackage{amsmath}
\newcommand{\bsb}{\boldsymbol}
\newcommand{\mb}{\mathbf}
\newcommand{\mc}{\mathcal}
\newcommand{\mexpc}{\mathrm{E}}
\newcommand{\ve}{\mathrm{vec}}
\newcommand{\toep}{\mathcal{T}}
\newcommand{\tr}{\mathrm{tr}}
\newcommand{\tred}{\color{red}}
\textheight=9.2in
\DeclareCaptionLabelFormat{lc}{\MakeLowercase{#1}~#2}
\captionsetup{labelfont=sc,labelformat=lc}
\renewcommand{\figurename}{Figure}
\renewcommand{\theequation}{R\arabic{equation}}
\usepackage[numbers]{natbib}

\setcounter{equation}{14}
\setcounter{figure}{1}
\setcounter{table}{1}


%\hyphenation{op-tical net-works semi-conduc-tor}
\begin{document}
The authors once again appreciate the helpful comments from the editor and both
reviewers.
Throughout this response letter, we will follow the same notational
conventions as our last response letter

\begin{center}
  {\LARGE \textbf{Authors' Response to Reviewer 1}}
\end{center}

% ~\citep[R][]{cheng2013thermal}
% \begin{align}
%   sfsdf
%   \label{eq:test}
% \end{align}
% \eqref{eq:test}
% \begin{figure}[!t]
%   \centering
%   \includegraphics[width=0.75\columnwidth]{./figs/test.eps}
%   \caption{Average throughput.}
%   \label{fig:test}
% \end{figure}
% Fig.~\ref{fig:test}

 
%%%%%%%%%%%%%%%%%%%%%%%%%%%%%%%%%%%%%%%%%%%%%%%%%%%%%%%%%%%%%%%%
\noindent
\emph{1. In their response letter, the authors have successfully addressed most
of my comments. However, they have not included in the paper the explanation
corresponding to the relationship between x variables and psi variables. Without
this explanation, I consider the paper is incomplete.}

\noindent \textbf{Authors' response:} 
Thank you for the comment. In addition to the definition of $\mathbf{x}^{(m)}$
from $\bm{\psi}^{(m)}$ in the first paragraph of Section III.B, in our revised
manuscript we have included immediately after Eq. (12) the explanation about
what is the mapping vector function $\bm{\psi}^{(m)}$ corresponding to a given
$\mathbf{x}^{(m)}\in\mathcal{S}$ from our last response letter. These two
sentences together should be sufficient to explain the the equivalence between
$\bm{\psi}^{(m)}$ and $\mathbf{x}^{(m)}$. We have also added a reference to [9]
where a similar technique of representing mapping function $\psi^{(m)}$ with a
2-D permutation matrix $\mathbf{x}^{(m)}$ is used, in order to further clarify
the relationship between $\mathbf{x}^{(m)}$ and $\bm{\psi}^{(m)}$.

We regret that in the revised manuscript due to the page limit we cannot provide
an example to demonstrate the relationship between $\mathbf{x}^{(m)}$
from $\bm{\psi}^{(m)}$ as the one included in our previous response letter.
\vspace{0.5cm}

%%%%%%%%%%%%%%%%%%%%%%%%%%%%%%%%%%%%%%%%%%%%%%%%%%%%%%%%%%%%%%%%
\noindent
\emph{2. On the other hand, in their response the authors claim that ``the
original problem (11) and (13) are equivalent and neither one is more complex
than the other''. The fact that two problems are equivalent does not mean they
have same complexity; e.g. there are many papers where a non-convex problem
becomes an equivalent convex problem, which is typically less complex.  Also,
you should at least provide any insight showing why (13) is as complex as (11),
given that it seems you have $Q^3$ variables for (13) and $2\times Q^2$ for
(11).}

\noindent \textbf{Authors' response:}
From our understanding, we presume that the reviewer is primarily concerned
about two types of complexity.

In terms of \textbf{algorithmic complexity}, the reviewer is definitely correct
in that two problems are equivalent does not mean they have the same
complexity~\citep[R][Sec. 4.1.3]{boyd2004convex}. For instance, the convex
equivalences of many non-convex problems can be efficiently solved using
standard convex optimization tools. In our case, there is no
evidence that either Eq.(11) or Eq.(13) leads to a solving algorithm with less
algorithmic complexity than the other.

The other aspect of complexity is the \textbf{notational complexity}, which is
what we believed the reviewer meant by claiming ``the authors transform the
original problem (11) into a more complex problem (13)'' in the second review.
Indeed Eq. (13) appears to have more variables than (11). In Eq. (11) we are
optimizing the mapping vector function for the $m$-th retransmission given all the mapping
vector functions for the previous retransmissions to maximize the approximated
BER, of which the variable space is $\{\psi_1^{(m)}[0], \ldots,
\psi_1^{(m)}[Q-1], \psi_2^{(m)}[0], \ldots, \psi_2^{(m)}[Q-1]\}$ whose
\textbf{dimensionality} is $2Q$. On the other hand, the dimensionality of the
variable space for Eq. (13) is indeed $Q^3$. However, an equivalent problem with
more varialbes does not necessarily leads to a more algorithmically complex
solution. For instance, in both Dantzig's simplex method for linear
programming with inequality constraints~\citep[R][]{gass1958linear} and Vapnik's
support vector machine in its hinge-loss function form~\citep[R][Eq.
(12.25)]{friedman2001elements}, additional slack variables are introduced to
facilitate easier solution.
  
More specifically, in our formulation, there is a
one-to-one mapping between $\bm{\psi}^{(m)}$ and $\mathbf{x}^{(m)}$, which
means the \textbf{cardinalities} of the feasible set for $\bm{\psi}^{(m)}$ in
Eq. (11) and $\mathbf{x}^{(m)}$ in Eq. (13) are the same. In fact, the
cardinalities of both $\Psi = \{\bm{\psi}^{(m)}| \psi_a^{(m)}[p]\in
\mathcal{C},\, \psi_a^{(m)}[p] \not= \psi_a^{(q)}; a=1,2; p,q =
0,\ldots,Q-1\}$ and $\mathcal{S}$ are $(Q!)^2$. As we adopt an iterative local
search (ILS) solution, searching in both $\Psi$ and $\mathcal{S}$ will result
in exactly the same algorithmic complexity.

In order to avoid the misconception of the variable space of Eq. (11), we have
added more context and references pointing to similar formulations in related
works in the first paragraph of Section III.B. We have also included the
constraints on $\bm{\psi}^{(m)}$ which define its feasible set in the first
paragraph of Section II. Moreover, as per the reviewer's suggestion, we have
explicitly stated in the last paragraph of Section III.B that both Eq. (11) and
Eq. (13) have a equally large search space of size $(Q!)^2$.

\vspace{0.5cm}

%%%%%%%%%%%%%%%%%%%%%%%%%%%%%%%%%%%%%%%%%%%%%%%%%%%%%%%%%%%%%%%%%
\noindent
\emph{3. I also thinks the use of the word "apparently" on page 2, line 13
should be avoided in that context.}

\noindent \textbf{Authors' response:}
Of course. We have removed this sentence in our revised manuscript. As explained
in our response to comment 1, we include a clearer explanation on the
relationship between $\mathbf{x}^{(m)}$ and $\bm{\psi}^{(m)}$ therefore it is
of little necessity to reiterate the meaning of the 2nd and 3rd dimension of
the 3-D matrix $\mathbf{x}^{(m)}$.


\vspace{0.5cm}


%%%%%%%%%%%%%%%%%%%%%%%%%%%%%%%%%%%%%%%%%%%%%%%%%%%%%%%%%%%%%%%%

Once again, we greatly appreciate the reviewer's comments and suggestions
that helped to further improve the completeness of our manuscript.

\newpage
\begin{center}
{\LARGE \textbf{Authors' Response to Reviewer 2}}
\end{center}

%%%%%%%%%%%%%%%%%%%%%%%%%%%%%%%%%%%%%%%%%%%%%%%%%%%%%%%%%%%%%%%%%%%%%
\noindent
\emph{I confirm my previous review and I suggest to accept the manuscript for
publication in IEEE communications Letters unaltered.}

\noindent \textbf{Authors' response:}
Once again we would like to extend our gratitude and appreciation for the
reviewer's recommendation and for affirming the contribution of this work.

\vspace{0.5cm}

 %\newpage
\bibliographystyle{IEEEtranN}
\bibliography{IEEEabrv,refs.bib}
\end{document} 