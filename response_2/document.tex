\documentclass[onecolumn, 11pt, draftclsnofoot]{IEEEtran}

\usepackage{epsfig,graphics,epsf,amsfonts,color,mathrsfs,amssymb,bm,caption,algorithm,algorithmic}
\usepackage{url}

\usepackage{amsmath}
\newcommand{\bsb}{\boldsymbol}
\newcommand{\mb}{\mathbf}
\newcommand{\mc}{\mathcal}
\newcommand{\mexpc}{\mathrm{E}}
\newcommand{\ve}{\mathrm{vec}}
\newcommand{\toep}{\mathcal{T}}
\newcommand{\tr}{\mathrm{tr}}
\newcommand{\tred}{\color{red}}
\textheight=9.2in
\DeclareCaptionLabelFormat{lc}{\MakeLowercase{#1}~#2}
\captionsetup{labelfont=sc,labelformat=lc}
\renewcommand{\figurename}{Figure}
\renewcommand{\theequation}{R\arabic{equation}}
\usepackage[numbers]{natbib}

\setcounter{equation}{14}
\setcounter{figure}{1}
\setcounter{table}{1}


%\hyphenation{op-tical net-works semi-conduc-tor}
\begin{document}
The authors once again appreciate the helpful comments from the editor and both
reviewers.
Throughout this response letter, we will follow the following notational
rules for citations and references:
\begin{itemize}
  \item Citations: references in this response letter are cited with prefix
  ``R'' (e.g. [R 1]) whereas references in the original manuscript are cited
  without prefix (e.g. [1]).
  \item Equations: equations in this response letter are referred to as
  ``Eq.~(R1)''  while those in the original manuscript as ``Eq.~(1)''.
  \item Figures: figures in this response letter are referred to as
  ``FIGURE~1''  while those in the original manuscript as ``Fig.~1''.
\end{itemize}
Also the reference numbers for equations, figures and tables
consistently continue to count up from their counterparts in our previous
response letter.

\begin{center}
  {\LARGE \textbf{Authors' Response to Reviewer 1}}
\end{center}

% ~\citep[R][]{cheng2013thermal}
% \begin{align}
%   sfsdf
%   \label{eq:test}
% \end{align}
% \eqref{eq:test}
% \begin{figure}[!t]
%   \centering
%   \includegraphics[width=0.75\columnwidth]{./figs/test.eps}
%   \caption{Average throughput.}
%   \label{fig:test}
% \end{figure}
% Fig.~\ref{fig:test}

 
%%%%%%%%%%%%%%%%%%%%%%%%%%%%%%%%%%%%%%%%%%%%%%%%%%%%%%%%%%%%%%%%
\noindent
\emph{1. However, the method (Section III.B) has still some points to be
clarify. The main concern is why the authors transform the original problem (11)
into a more complex problem (13). The assignment variables x play a key role
here, but the authors do not explain what they are modeling with the second and
third component of x; also, in the expression of x, the meaning of indexes i and
j is not revealed.}

\noindent \textbf{Authors' response:} 

\vspace{0.5cm}

%%%%%%%%%%%%%%%%%%%%%%%%%%%%%%%%%%%%%%%%%%%%%%%%%%%%%%%%%%%%%%%%
\noindent
\emph{2. the original formulation of x is more clear than using Kronecker delta.
}

\noindent \textbf{Authors' response:}

\vspace{0.5cm}

%%%%%%%%%%%%%%%%%%%%%%%%%%%%%%%%%%%%%%%%%%%%%%%%%%%%%%%%%%%%%%%%%
\noindent
\emph{3. what do you mean with ``$\psi_0$ represents Gary mapping'' (page 2,
column 2, line 13)}

\noindent \textbf{Authors' response:}


\vspace{0.5cm}

%%%%%%%%%%%%%%%%%%%%%%%%%%%%%%%%%%%%%%%%%%%%%%%%%%%%%%%%%%%%%%%%%%
\noindent
\emph{4. why $\lambda = 1/(4\sigma^2)$ in the proof of proposition 1?}

\noindent \textbf{Authors' response:}

 
\vspace{0.5cm}

%%%%%%%%%%%%%%%%%%%%%%%%%%%%%%%%%%%%%%%%%%%%%%%%%%%%%%%%%%%%%%%%

Once again, we thank the reviewer for the review and suggestions that helped to
significantly improve our revised manuscript. 

\newpage
\begin{center}
{\LARGE \textbf{Authors' Response to Reviewer 2}}
\end{center}

%%%%%%%%%%%%%%%%%%%%%%%%%%%%%%%%%%%%%%%%%%%%%%%%%%%%%%%%%%%%%%%%%%%%%
\noindent
\emph{The application scenario investigated by the authors is timely and of
great interest. This reviewer appreciates the efforts the authors have devoted
to revise the paper and thinks it is worth to publish the proposed approach and
results based on it on IEEE communications letters.}

\noindent \textbf{Authors' response:}
Thank you for your recommendation and for affirming the contribution of this work. 

\vspace{0.5cm}

 %\newpage
\bibliographystyle{IEEEtranN}
\bibliography{IEEEabrv,refs.bib}
\end{document} 